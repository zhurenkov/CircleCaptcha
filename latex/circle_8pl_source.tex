\hypertarget{circle_8pl_source}{}\section{circle.\+pl}

\begin{DoxyCode}
00001 #!/usr/bin/perl
00002 use strict;
00003 use warnings;
00004 use POSIX;
00005 use SVG;
00006 
00007 open CAP, ">cap.svg" or die "Невозможно создать SVG-файл: $!\(\backslash\)n";
00008  
00009 our $m=100;
00010 # create an SVG object with a size of 200x200 pixels
00011 # Создаём SVG объект размером 200 на 200 пикселей.
00012 my $svg = SVG->new(
00013 width => $m*2,
00014 height => $m*2,
00015 );
00016 
00017 our $n=shift()+1;
00018 my $r;
00019 my %rgb;
00020 
00021 # add a circles
00022 # Создаём случайные круги.
00023 for(my $i=0;$i<$n;$i++)\{
00024 $r=10+floor(rand()*($m-10));
00025 %rgb=(R=>floor(rand()*255), G=>floor(rand()*255), B=>floor(rand()*255));
00026 
00027 $svg->circle(
00028 r => $r,
00029 cx => $r+floor(rand()*2*($m-$r)),
00030 cy => $r+floor(rand()*2*($m-$r)),
00031 style => \{
00032  'fill' => 'rgb('.$rgb\{R\}.','.$rgb\{G\}.','.$rgb\{B\}.')',
00033  'stroke' => 'black',
00034  'stroke-width' => 1,
00035  'stroke-opacity' => .5,
00036  'fill-opacity' => .5,
00037 \},
00038 );
00039 print $r." ";
00040 print $rgb\{R\}."+".$rgb\{G\}."+".$rgb\{B\}."; ";
00041 \}
00042 
00043 # now render the SVG object, implicitly use svg namespace
00044 # Рендерим SVG объект и превращаем его в svg-файл.
00045 print CAP $svg->xmlify;
\end{DoxyCode}
